\documentclass[12pt,t]{beamer}
\usepackage{graphicx}
\setbeameroption{hide notes}
\setbeamertemplate{note page}[plain]
\usepackage{listings}

% header.tex: boring LaTeX/Beamer details + macros

% get rid of junk
\usetheme{default}
\beamertemplatenavigationsymbolsempty
\hypersetup{pdfpagemode=UseNone} % don't show bookmarks on initial view


% font
\usepackage{fontspec}
\setsansfont
  [ ExternalLocation = fonts/ ,
    UprightFont = *-regular ,
    BoldFont = *-bold ,
    ItalicFont = *-italic ,
    BoldItalicFont = *-bolditalic ]{texgyreheros}
\setbeamerfont{note page}{family*=pplx,size=\footnotesize} % Palatino for notes
% "TeX Gyre Heros can be used as a replacement for Helvetica"
% I've placed them in fonts/; alternatively you can install them
% permanently on your system as follows:
%     Download http://www.gust.org.pl/projects/e-foundry/tex-gyre/heros/qhv2.004otf.zip
%     In Unix, unzip it into ~/.fonts
%     In Mac, unzip it, double-click the .otf files, and install using "FontBook"

% named colors
\definecolor{offwhite}{RGB}{0,5,15}
\definecolor{gray}{RGB}{100,100,100}

\ifx\notescolors\undefined % slides
  \definecolor{foreground}{RGB}{0,0,0}
  \definecolor{background}{RGB}{231,231,231}
  \definecolor{title}{RGB}{148,81,41}
  \definecolor{subtitle}{RGB}{128,128,41}
  \definecolor{hilit}{RGB}{153,0,51}
  \definecolor{vhilit}{RGB}{0,143,48}
  \definecolor{lolit}{RGB}{100,100,100}
  \definecolor{myyellow}{rgb}{1,1,0.7}
\else % notes
  \definecolor{background}{RGB}{0,0,0}
  \definecolor{foreground}{RGB}{231,231,231}
  \definecolor{title}{RGB}{228,160,120}
  \definecolor{subtitle}{RGB}{233,83,131}
  \definecolor{hilit}{RGB}{133,255,128}
  \definecolor{vhilit}{RGB}{0,255,128}
  \definecolor{lolit}{RGB}{160,160,160}
\fi
\definecolor{nhilit}{RGB}{128,255,128}  % hilit color in notes
\definecolor{nvhilit}{RGB}{0,255,128} % vhilit for notes

\newcommand{\hilit}{\color{hilit}}
\newcommand{\vhilit}{\color{vhilit}}
\newcommand{\nhilit}{\color{nhilit}}
\newcommand{\nvhilit}{\color{nvhilit}}
\newcommand{\lolit}{\color{lolit}}

% use those colors
\setbeamercolor{titlelike}{fg=title}
\setbeamercolor{subtitle}{fg=subtitle}
\setbeamercolor{institute}{fg=lolit}
\setbeamercolor{normal text}{fg=foreground,bg=background}
\setbeamercolor{item}{fg=foreground} % color of bullets
\setbeamercolor{subitem}{fg=lolit}
\setbeamercolor{itemize/enumerate subbody}{fg=lolit}
\setbeamertemplate{itemize subitem}{{\textendash}}
\setbeamerfont{itemize/enumerate subbody}{size=\footnotesize}
\setbeamerfont{itemize/enumerate subitem}{size=\footnotesize}

% page number
\setbeamertemplate{footline}{%
    \raisebox{5pt}{\makebox[\paperwidth]{\hfill\makebox[20pt]{\lolit
          \scriptsize\insertframenumber}}}\hspace*{5pt}}

% add a bit of space at the top of the notes page
\addtobeamertemplate{note page}{\setlength{\parskip}{12pt}}

% default link color
\hypersetup{colorlinks, urlcolor={hilit}}

\ifx\notescolors\undefined % slides
  % set up listing environment
  \lstset{language=bash,
          basicstyle=\ttfamily\scriptsize,
          frame=single,
          commentstyle=,
          backgroundcolor=\color{darkgray},
          showspaces=false,
          showstringspaces=false
          }
\else % notes
  \lstset{language=bash,
          basicstyle=\ttfamily\scriptsize,
          frame=single,
          commentstyle=,
          backgroundcolor=\color{offwhite},
          showspaces=false,
          showstringspaces=false
          }
\fi

% a few macros
\newcommand{\bi}{\begin{itemize}}
\newcommand{\bbi}{\vspace{24pt} \begin{itemize} \itemsep8pt}
\newcommand{\ei}{\end{itemize}}
\newcommand{\ig}{\includegraphics}
\newcommand{\subt}[1]{{\footnotesize \color{subtitle} {#1}}}
\newcommand{\ttsm}{\tt \small}
\newcommand{\ttfn}{\tt \footnotesize}
\newcommand{\figh}[2]{\centerline{\includegraphics[height=#2\textheight]{#1}}}
\newcommand{\figw}[2]{\centerline{\includegraphics[width=#2\textwidth]{#1}}}


%%%%%%%%%%%%%%%%%%%%%%%%%%%%%%%%%%%%%%%%%%%%%%%%%%%%%%%%%%%%%%%%%%%%%%
% end of header
%%%%%%%%%%%%%%%%%%%%%%%%%%%%%%%%%%%%%%%%%%%%%%%%%%%%%%%%%%%%%%%%%%%%%%

% title info
\title{Electrical Principles - Part 1}
\subtitle{Storage and transfer of electrical energy}
\author{\href{www.drpineda.ca}{Mario Pineda}}
\institute{Queen Elizabeth High School}
\date{\href{www.drpineda.ca}{\tt \scriptsize \color{foreground} drpineda.ca}
\\[-4pt]
\href{https://github.com/mariopineda}{\tt \scriptsize \color{foreground} github.com/mariopineda}
\\[-4pt]
\href{https://twitter.com/therocsci}{\tt \scriptsize \color{foreground} @therocsci}
\\[2pt]
\scriptsize {\lolit Slides:} \href{https://github.com/mariopineda/electrical-principles-slides}{\tt \scriptsize
  \color{foreground} github.com/mariopineda/electrical-principles-slides}
}

\begin{document}

% title slide
{
\setbeamertemplate{footline}{} % no page number here
\frame{
  \titlepage
  \hfill \includegraphics[height=6mm]{images/cc-zero.png}
  \note{}
}

\begin{frame}[c]{Objectives}
\bbi
\item Describe the major features of cells
\item Know the two major types of cells
\ei
\end{frame}

\begin{frame}[c]{Battery Classification}
\bbi
\item Primary Cells: Cannot be recharged as the chemical reactions cannot be reversed.
\item Secondary Cells: Can be recharged by passing a current through the cell in the opposite direction.
\ei
\end{frame}

\begin{frame}[c]{Battery Chemistries: Lithium Polymer (LiPo)}
\bbi
\item Type: Secondary cell
\item Chemistry: Lithium-ion and polymer (gel) electrolyte
\item Pros: High energy content, light, can be produced in any shape
\item Cons: Physical damage, over charge or too high temperature can cause cells to fail catastrophically (fire, explosion)
\item Voltage: 2.7-4.2V
\item Usage: personal electronics (laptops, cell phones), RC vehicles, International Space Station (since 2017)
\item Invented by Sony in 1991
\ei
\end{frame}

\begin{frame}[c]{Battery Chemistries: Lithium Iron Phosphate (LiFePO4 aka LFP)}
\bbi
\item Type: Secondary cell
\item Chemistry: 
\item Pros: Constant discharge voltage, stable and safe (compare to LiPo), many recharge cycles
\item Cons: Lower energy density 
\item Voltage: 3.2V per cell
\item Usage: Electric vehicles, power tools, RC vehicles
\ei
\end{frame}

\begin{frame}[c]{Battery Chemistries: Lead-Acid Battery}
\bi
\item Type: Secondary cell
\item Chemistry: Lead, lead dioxide, electrolyte concentrated sulfuric acid
\item Pros: High energy density, many recharge cycles, cheap
\item Cons: Effectiveness reduced at low temperatures, self-discharges, contains lead and concentrated sulfuric acid
\item Voltage: 2V per cell
\item Usage: Vehicle starter and ignition, backup power supplies (computer UPS units)
\item Oldest type of rechargable 
\ei
\end{frame}

\begin{frame}[c]{Battery Chemistries: NiCd }
\bi
\item Type: Secondary cell
\item Chemistry:
\item Pros: Long Shelf life,many recharge cycles, good performance at low temperatures, can produce very large instantaneous currents (1000-8000A for a second) 
\item Cons: Contains Cadmium (heavy, expensive and toxic), corrosive electrolytes
\item Voltage: 1.2V per cell
\item Usage: Portable electronic equipment, e.g. flash lights, aircraft and satellite systems, starting large disel engines and turbines.
\ei
\end{frame}

\begin{frame}[c]{Battery Chemistries: Nickel-metal Hydride (NiMH)}
\bi
\item Type: Secondary cell
\item Chemistry: 
\item Pros:
\item Cons: Overcharging can rupture 
\item Voltage: 1.4-1.6V
\item Usage: Consumer electronics
\ei
\end{frame}

\begin{frame}[c]{Fuel Cells}
\bi
\item Fuel and oxidant are continuously supplied from the outise while vaste products are removed.
\item Pros: High-efficiency of energy conversion, no harmful waste products, no need to recharge
\item Cons: Expensive, fuel gases must be stored in high-pressure tanks, moderate power output
\item Use: The space shuttle and Apollo space program, military submarines
\ei
\end{frame}
\end{document}
